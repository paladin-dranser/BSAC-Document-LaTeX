%%% Пункт 3.1 — фармат А4 (210x297 мм);
%%%           — памер асноўнага шрыфту 12 пунктаў;
\documentclass[a4paper,12pt]{extarticle} % extarticle — падтрымка шрыфтаў з памерам 8pt, 9pt, 10pt, 11pt, 12pt, 14pt, 17pt, 20pt

%%% Пункт 3.1 — памер палёў старонкі (малюнак 3.1, СТП 01-2017)
\usepackage{geometry}
    \geometry{left=30mm}
    \geometry{right=10mm}
    \geometry{top=20mm}
    \geometry{bottom=20mm}

%%% Пункт 3.1 — шрыфт Times New Roman
\usepackage{fontspec}
    \setmainfont{Times New Roman}

%%% Падключэнне беларускай мовы
\usepackage[english, belarusian]{babel}

%%% Дазвол на кірыліцу ў формулах
\usepackage{mathtext}

%%% Для ўключэння pdf дакументаў у канчатковы файл
\usepackage{pdfpages}

%%% Пераносы ў словах са злучком
%%% Злучок у словах замяняецца на \hyph: апаратна\hyphпраграмны
%%% https://stackoverflow.com/questions/2193307/how-to-get-latex-to-hyphenate-a-word-that-contains-a-dash
\def\hyph{-\penalty0\hskip0pt\relax}

%%% мадыфікатары напісання
\usepackage{soul}

%%% спасылкі ў pdf
\usepackage{hyperref}

%%% Пункт 3.1 — інтэрвал паміж радкамі
%%%             мае складаць 1,5 машынапіснага інтэрвалу
\usepackage{setspace}
    \onehalfspacing

%%% «Разумная» коска
\usepackage{icomma}

%%% перанос знакаў у формулах (па Львоўскаму)
\newcommand*{\hm}[1]{#1\nobreak\discretionary{}
{\hbox{$\mathsurround=0pt #1$}}{}}

%%% подпісы
\usepackage{caption}
    %%% Пункт 3.3 — подпіс табліцы
    \captionsetup[table]{
        name=Табліца,
        labelsep=endash,
        justification=raggedright,
        singlelinecheck=false
    }
    %%% Пункт 3.5 — подпіс ілюстрацый
    \captionsetup[figure]{
        name=Малюнак,
        labelsep=endash,
        justification=centering
    }

\usepackage{fancyhdr} % пакет для устаноўкі калонтытулаў
    %%% Стыль старонкі (раздзел)
    %%% Пункт 3.2 — нумарацыя старонак у ніжнім правым кутку
    %%%           — (малюнак 3.1, СТП 01-2017)
    \fancypagestyle{section}{
        \fancyhf{} % ачыстка стылю
        \renewcommand{\headrulewidth}{0pt} % прыбраць падзяляльную лінію ўверсе старонкі
        \renewcommand{\footrulewidth}{0pt} % прыбраць падзяляльную лінію ўнізе старонкі
        \rfoot{\thepage} % нумарацыя ў ніжнім правым кутку
    }

%%% Водступ першага абзаца ў section (адпаведна з астатнімі абзацамі)
\usepackage{indentfirst}

%%% адключэнне змянення прабелаў паміж словамі/сказамі
\frenchspacing

%%% Пункт 3.1 — велічыня водступу абзаца 1,25 см
\setlength{\parindent}{1.25cm}

%%% магчымасць адключаць пераносы слоў
\usepackage{hyphenat}

%%% Рэдагаваць выгляд загалоўкаў
\usepackage{titlesec}
%%% Пункт 3.1 — загалоўкі раздзелаў пішуцца вялікімі літарамі
%%%           — водступ абзаца
%%%           — паўтоўсты шрыфт Times New Roman 14 памеру
%%%           — перанос слоў не дапускаецца
%%%           — кожны раздзел мае пачынацца з новай старонкі
%%%           — адлегласць паміж загалоўкам і тэкстам у 1 прабельны радок
%%%           — адлегласць паміж загалоўкам і падзагалоўкам у 1 прабельны радок
%%%           — дадатак Р
    \titleformat{\section}
        {\fontsize{14}{14} \bfseries \raggedright \newpage}
        {}{0pt}{\hspace{1.25cm}\thesection~\MakeUppercase} 
    \titlespacing{\section}{0pt}{0pt}{1.5em}
%%% Пункт 3.1 — паўтоўсты шрыфт Times New Roman з памерам асноўнага тэксту
%%%           — адлегласць паміж падзагалоўкам і тэкстам у 1 прабельны радок
%%%           — дадатак Р
    \titleformat{\subsection}
        {\fontsize{12}{12} \bfseries \raggedright}
        {}{0pt}{\hspace{1.25cm}\thesubsection~} 
    \titlespacing{\subsection}{0pt}{1.5em}{1.5em}
