%%% Пункт 3.1 — фармат А4 (210x297 мм);
%%%           — памер шрыфта 12 пунктаў;
\documentclass[a4paper,12pt]{extarticle} % extarticle — падтрымка шрыфтаў з памерам 8pt, 9pt, 10pt, 11pt, 12pt, 14pt, 17pt, 20pt

%%% Пункт 3.1 — памер палёў (малюнак 3.1, СТП 01-2017)
\usepackage{geometry}
    \geometry{left=30mm}
    \geometry{right=10mm}
    \geometry{top=20mm}
    \geometry{bottom=20mm}

%%% Пункт 3.1 — шрыфт Times New Roman
\usepackage{fontspec}
    \setmainfont{Times New Roman}

%%% Падключэнне беларускай мовы
\usepackage[english, belarusian]{babel}

%%% Дазвол на кірыліцу ў формулах
\usepackage{mathtext}

%%% Для ўключэння pdf дакументаў у канчатковы файл
\usepackage{pdfpages}

%%% Пераносы ў словах са злучком
%%% Злучок у словах замяняецца на \hyph: апаратна\hyphпраграмны
%%% https://stackoverflow.com/questions/2193307/how-to-get-latex-to-hyphenate-a-word-that-contains-a-dash
\def\hyph{-\penalty0\hskip0pt\relax}

%%% мадыфікатары напісання
\usepackage{soul}

%%% спасылкі ў pdf
\usepackage{hyperref}

%%% Пункт 3.1 — інтэрвал паміж радкамі
%%%             мае складаць 1,5 машынапіснага інтэрвалу
\usepackage{setspace}
    \onehalfspacing

%%% «Разумная» коска
\usepackage{icomma}

%%% перанос знакаў у формулах (па Львоўскаму)
\newcommand*{\hm}[1]{#1\nobreak\discretionary{}
{\hbox{$\mathsurround=0pt #1$}}{}}

%%% Пункт 3.3 — подпіс табліцы;
%%% Пункт 3.5 — подпіс ілюстрацый
\usepackage{caption}
\captionsetup[table]{name=Табліца,labelsep=endash, justification=raggedright, singlelinecheck=false}
\captionsetup[figure]{name=Малюнак,labelsep=endash, justification=centering}

%%% Пункт 3.2 — нумарацыя старонак у ніжнім правым кутку
%%%           — (малюнак 3.1, СТП 01-2017)
\usepackage{fancyhdr} % пакет для устаноўкі калонтытулаў
    \pagestyle{fancy} % зменя стылю афармлення старонак
    \renewcommand{\headrulewidth}{0pt} % прыбраць падзяляльную лінію ўверсе старонкі
    \renewcommand{\footrulewidth}{0pt} % прыбраць падзяляльную лінію ўнізе старонкі
    \fancypagestyle{plain}{
        \fancyhf{} % ачыстка бягучага стылю
        \rfoot{\thepage} % нумарацыя ў ніжнім прывым кутку
    }

%%% Водступ першага абзаца ў section (адпаведна з астатнімі азбацамі)
\usepackage{indentfirst}

%%% Пункт 3.1 — велічыня водступу абзаца 1,25 см
\setlength{\parindent}{1.25cm}

%%% Пункт 3.1 — водступ абзаца 1,25 см для section
\makeatletter
\renewcommand\@seccntformat[1]{\hspace*{\parindent}\csname the#1\endcsname\quad}
\makeatother
